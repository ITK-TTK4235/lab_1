\section{Oppgave 1 (100\%) - Funksjonsblokkdiagram}\label{sec:3-oppgave1}

Denne oppgaven skal gi dere grunnleggende kunnskaper om programmering av PLSer med \verb|funksjonsblokkdiagrammer|. Før dere spør vitass eller læringsassistenter, sjekk ut appendiks \ref{app:feilsøking} for vanlige feil som kan oppstå under labben.

\subsection{Grunnleggende funksjonsprogrammering av heis}

På hver labplass ligger det en manual med en detaljert veiledning for å komme i gang med programmering på PLSer fra Siemens: “Working with STEP 7”. Vi vil her ta utgangspunkt i de første 4 kapitlene, men siden manualen gjennomgår et eksempel av et annet system enn heisen
vil vi gjøre noen endringer underveis. Til å starte, skal dere se på følgende:

\begin{itemize}
        \item I kapittel 2 må dere velge riktig hardware. Typebetegnelsen (\verb|CPU313 / CPU315-| 
        \verb|2 DP / andre|) på modulene står øverst i høyre hjørne på den fysiske PLS-blokken fra figur \ref{fig:siemens-pls}. Vi vil bruke “\verb|Function Block Diagram|” (\verb|FBD|) som
programmeringsspråk.


\item Vi må nå fylle ut symboltabellen, slik at vi kan bruke mer lesbare navn enn hardwareadressene.
Se i kapittel 3 for hvordan man redigerer symboltabellen. Adressene til
knapper, lys og motor er gitt tidligere i subseksjon \ref{subsec:intefrace}. I kapittel 4.1 skal dere altså ikke
kopiere symboltabell fra eksempel-prosjektet.

\item I kapittel 4 trenger dere ikke å se på 4.2 og 4.3 som omhandler programmering i \verb|Statement List| (\verb|STL|) eller \verb|Ladder Logic| (\verb|LAD|). Men prøv gjerne å se hvordan programmet deres ville
ha sett ut i de andre språkene ved hjelp av menyen \verb|View-> LAD|, \verb|View-> FBD| eller \verb|View-> STL|.
\item I stedet for eksempelprogrammet i kapittel 4.4 skal vi skrive et enkelt program som styrer
heisen manuelt med opp- og ned-knappene: Når minst en av “Jeg-vil-ned-knappene”
trykkes, skal heisen kjøres nedover. Tilsvarende skal heisen kjøre oppover når minst en av
“Jeg-vil-opp-knappene” trykkes. Når ingen av opp- eller ned-knappene holdes nede skal
heisen stå i ro.
\end{itemize}

\subsubsection*{Tips 1}
Det er lurt å gruppere programdelene tidlig ettersom programmet dere lager vil bli forholdsvis stort. Lag derfor (minst) en funksjon (\verb|FC|) for hver deloppgave. Dere må huske å kalle funksjonene fra \verb|OB1|. \verb|OB1| er en organisasjonsblokk som kjøres
syklisk i PLSen og er den eneste organisasjonsblokken vi trenger i hele denne lab-oppgaven.
Funksjonsblokker (\verb|FB|er) og funksjoner (\verb|FC|er) ligger på nivået under organisasjonsblokkene
i programhierarkiet. \verb|FB|er og \verb|FC|er må kalles fra en annen blokk for at koden skal bli utført.
\verb|FB|er har en datablokk (\verb|DB|) knyttet til seg mens \verb|FC|er ikke har det. Hvis man ikke trenger å lagre unna data er det derfor best og enklest å bruke en \verb|FC| (funksjon), noe som er tilfelle i alle
oppgavene for dag 1.

\subsubsection*{Tips 2}
Det er anbefalt å styre motoren med to “nettverk”: ett som setter retningen på motoren, og ett som
setter hastigheten. For å sette ut analogt signal til motoren kan dere bruke to \verb|MOVE|-blokker.
La en \verb|MOVE|-blokk sette et passende stort signal til motoren når en knapp er trykket og en
annen \verb|MOVE|-blokk sette ut 0 når ingen knapper er trykket. For å styre retningen kan det være
lurt å bruke en \verb|SR|-vippe.

\subsection{Bestillingsknapplys funksjonalitet}

Gjør følgende utvidelse av programmet: Registrer hvilken etasje heisen sist
passerte og sett ut lys i tilsvarende lysdiode på betjeningstavla for heisen. Lysdioden skal lyse
helt til heisen når en ny etasje.

\subsection{Heisdør og obstruksjonsføler funksjonalitet}

Gjør følgende utvidelse av programmet: Når noen utløser obstruksjonsføleren i
døra skal døra åpnes og holdes åpen i 3 sekunder etter at signalet fra obstruksjonsføleren har
gått lavt. Deretter skal døra lukkes.

\subsubsection*{Tips 3}
Se beskrivelsene av timer-instruksjonene (velg aktuell timer og tast "\verb|F1|"). Dere må selv avgjøre
hvilken timer som er best egnet for formålet. Se også appendiks \ref{app:timer}.

\subsection{Viderekommende funksjonsprogrammering av heis}

Implementer en forenklet algoritme for personheisen:

\begin{itemize}
    \item Stopp heisen og åpne døra i 3 sekunder i hver etasje.
    \item Etter 3 sekunder kjører heisen videre i samme retning til en ny etasje nås.
    \item Når heisen har nådd toppetasjen, settes retningen til nedover. 
    \item Når heisen har nådd bunnetasjen settes retningen til oppover. 
    \item Bestillingsknappene skal ikke ha noen virkning i denne forenklede implementeringen.
    \item Ta med registereringen av siste passerte etasje som dere gjorde i oppgave 1a.
\end{itemize}

\subsection*{\color{RWTHrot100}Lagring}
Det kan lønne seg å ta vare på programmet i de foregående punktene da deler av det kan brukes
senere i oppgaven. Trenger dere å ta med dere prosjektet kan dere enten kopiere alle filene
STEP 7 lager eller dere kan lagre prosjektet komprimert. I sistnevnte tilfelle brukes menyvalget \verb|File->Archive|. For å kunne jobbe med prosjektet senere må dere pakke det ut med menyvalget
\verb|File->Retrieve|.


\section{Oppgave 2 - Sekvensielle funksjonsdiagram (Frivillig)}\label{sec:4-oppgave}

Denne oppgaven skal gi dere grunnleggende kunnskaper om programmering av PLSer med \verb|sekvensielle funksjonsblokkdiagrammer| (\verb|SFC|). Her skal vi bruke verktøyet "S7 Graph"

\subsection{Grunnleggende sekvensprogrammering av heis}

Ta utgangspunkt i kapittel 3 fra manualen “S7 Graph V5.2 for S7-300/400 - Programming
Sequential Control Systems” for å implementere den forenklede personheisen fra oppgave 1c, men nå ved bruk av sekvensprogrammering (S7 GRAPH). Dere skal altså ikke taste inn drilleksempelet
fra manualen, men bare bruke dette som et eksempel på hvordan et slikt program
kan se ut. Fortsett på prosjektet fra dag 1: Legg til en ny funksjonsblokk (Se kapittel 5.1 i
"Working with STEP 7"), og velg \verb|GRAPH| som programmeringsspråk.


Før dere laster ned programmet til den fysiske PLSen må dere gjøre en liten endring i oppsettet
til funksjonsblokken som inneholder sekvensprogrammet: I “\verb|GRAPH|”-editoren velger dere:
\texttt{Options->Block Settings->Compile/Save->Executability->}\newline\verb|Full code|. Hele koden for
sekvensen dere har programmert samt en del systemspesifikke ting blir da lagt inn i
funksjonsblokken (\verb|FB|en) og den tilhørende datablokken (\verb|DB|en). 

Hvis systemet allerede har
opprettet en blokk som heter \verb|FC72|, må dere slette den fra prosjektet deres. \verb|FC72| inneholder en
del generelle ting for administrering av sekvenser. Dersom vi kun har en sekvensblokk vil vi
spare minneplass hvis vi genererer “full kode” i den aktuelle funksjonsblokka. PLSene vi har
på labben har så begrenset minnekapasitet at vi kun har plass til en sekvensblokk og denne må
være generert som “full kode”.

\subsubsection*{Tips 4}

Det går ikke an å kalle \verb|MOVE|-blokker fra S7 GRAPH, vi kan kun sette verdien av utganger
eller boolske variabler. Hvis man legger til en ny boolsk variabel i symboltabellen (som for eksempel: "\verb|MOTOR_ENABLE|"), så kan dere bruke denne som inngang til \verb|MOVE|-blokker i en ny
funksjon (\verb|fc|). Se i appendiks \ref{app:adresse} for hvordan man oppretter variabler i lokalt minne.

\subsection{Bestillingsknapplys funksjonalitet}

Utvid funksjonaliteten slik at når noen trykker på en bestillingsknapp lyser knappen helt til bestillingen er effektuert.


\subsubsection*{Tips 5}
Det kan lønne seg å sette lyset i bestillingsknappene med en \verb|SR|-vippe eller en “Sett-operasjon” i et \verb|funksjonsblokkdiagram|, mens resetting av lyset i bestillingsknappene i et \verb|sekvensielt funksjonsdiagram|.

\subsection{Viderekommende sekvensprogrammering av heis}

\begin{itemize}
    \item Forbedre algoritmen slik at heisen stopper i 2. og 3. etasje bare hvis det er noen som har bestilt den derfra. Når heisen kommer til endeetasjene (1. og 4.) skal den fortsatt alltid stoppe.

\item Gjør heisen mer effektiv ved å la være å stoppe i midtetasjene hvis det er mottatt en
bestilling i motsatt retning av hva heisen går i for øyeblikket. Det er for eksempel ingen vits
i å stoppe i 3. etasje hvis heisen er på vei oppover og personene i 3. etasje vil ned.
\end{itemize}


